\subsection{Timed Input/Output Automata}\label{sec:TIOA}
\emph{Timed Input/Output Automata}, abbreviated to TIOA, are basic automata with two extensions \cite{ecdartheory}. One of them being the notion of time. Time is tracked using clock variables, which can be used to make conditions on the system being in a location or performing an action. The second is having actions classified as either input or output actions. Input actions triggers from the outside, where output actions triggers from within. 

The formal definition of a TIOA is a tuple \cite{ecdartheory} $$(Loc, I_{0}, Act, Clk, E, Inv)$$  
where:

\begin{itemize}
    \item $Loc$ is a finite set of locations.
    \item $i_{0}$ is the initial location, so $i_{0} \in Loc$.
    \item $Act$ is a finite set of actions partitioned into inputs ($Act_{i}$) and outputs ($Act_{o}$).
    \item $Clk$ is a finite set of clocks.
    \item $E \subseteq Loc \times Act \times \mathcal{G} \times \mathcal{U} \times Loc$ is a set of edges.
    \item $Inv : Loc \mapsto \mathcal{B}(Clk)$ is a location invariant function. 
\end{itemize}

\subsection{Timed Input/Output Transistion Systems}\label{sec:TIOTS}

\emph{Timed Input/Output Transistion System}, TIOTS, is the semantic representation of TIOA and is used to analyze them, to ensure their correctness. TIOA are finite representations of TIOTS \cite{}. A TIOTS is a tuple \cite{}  $$(Q, q_{0}, Act, \rightarrow)$$

where:

\begin{itemize}
    \item $Q$ is a possibly infinite set of states.
    \item $q_{0}$ is the initial location, so $q_{0} \in Q$.
    \item $Act$ is a finite set of actions partitioned into inputs ($Act_{i}$) and outputs ($Act_{o}$).
    \item $\rightarrow \subseteq Q \times (Act \times \mathbb{R}_{\geq 0}) \times Q$ is a transition relation.
\end{itemize}

\subsubsection{Specification}
A TIOTS is a specification if all of it states are input enabled, this means that if $\forall i? \in Act_i : \exists q' \in Q$ which means that each state in a TIOA should be able to take input, at any time. i.e. $q \rightarrow^{i?} q'$\cite{}
\emph{Timed Input/Output Automata}, abbreviated to TIOA, are basic automata with two extensions \cite{ecdartheory}. One of them being the notion of time. Time is tracked using clock variables, which can be used to make conditions on the system being in a location or performing an action. The second is having actions classified as either input or output actions. Input actions triggers from the outside, where output actions triggers from within. 

The formal definition of a TIOA is a tuple \cite{ecdartheory} $$(Loc, I_{0}, Act, Clk, E, Inv)$$  
where:

\begin{itemize}
    \item $Loc$ is a finite set of locations.
    \item $i_{0}$ is the initial location, so $i_{0} \in Loc$.
    \item $Act$ is a finite set of actions partitioned into inputs ($Act_{i}$) and outputs ($Act_{o}$).
    \item $Clk$ is a finite set of clocks.
    \item $E \subseteq Loc \times Act \times \mathcal{G} \times \mathcal{U} \times Loc$ is a set of edges.
    \item $Inv : Loc \mapsto \mathcal{B}(Clk)$ is a location invariant function. 
\end{itemize}

\subsection{Timed Input/Output Transistion Systems}\label{sec:TIOTS}

\emph{Timed Input/Output Transistion System}, TIOTS, is the semantic representation of TIOA and is used to analyze them, to ensure their correctness. TIOA are finite representations of TIOTS \cite{}. A TIOTS is a tuple \cite{}  $$(Q, q_{0}, Act, \rightarrow)$$

where:

\begin{itemize}
    \item $Q$ is a possibly infinite set of states.
    \item $q_{0}$ is the initial location, so $q_{0} \in Q$.
    \item $Act$ is a finite set of actions partitioned into inputs ($Act_{i}$) and outputs ($Act_{o}$).
    \item $\rightarrow \subseteq Q \times (Act \times \mathbb{R}_{\geq 0}) \times Q$ is a transition relation.
\end{itemize}

\subsubsection{Specification}
A TIOTS is a specification if all of it states are input enabled, this means that if $\forall i? \in Act_i : \exists q' \in Q$ which means that each state in a TIOA should be able to take input, at any time. i.e. $q \rightarrow^{i?} q'$\cite{}
