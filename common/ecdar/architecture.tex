\subsection{Architecture}\label{sub:architecture}
ECDAR consists of four major systems as seen in \autoref{fig:ECDAR-architecture};
two engines: Reveaal and j-Ecdar, The ECDAR GUI, and a test framework.
The GUI and the test framework communicate with the engines through Protocol Buffers (Protobuf) and gRPC (\textbf{G}oogle \textbf{R}emote \textbf{P}rocedure \textbf{C}all) as indicated by the arrows.
\begin{figure}[H]
    \centering
    \includegraphics[width=0.75\textwidth]{common/figures/ArchOverview.png}
    \caption{The architecture of ECDAR visualized \cite{ECDARNET}.}
    \label{fig:ECDAR-architecture}
\end{figure}
% Figure old, get new

The graphical user interface (\textbf{GUI}) is written in JavaFX \cite{ECDARNET}  which is a graphics and media library for Java. 
The interface provides a set of tools that enable the user to model real time systems. 
The GUI sends queries to the engines through gRPC using Protobuf. 

The gRPC and Protobuf frameworks have the advantage of being cross platform and language independent \cite{gRPC}\cite{google_protocol_nodate}, simplifying the integration process and making them easy to use.

ECDAR's has two verification engines, j-Ecdar and Reveaal, to ensure reliability. 
With two engines, their results can be compared which will help ensure correctness in both engines.
The j-Ecdar component is an engine written in Java.
The main priorities for this engine are readability and correctness, and, as stated on ECDAR's homepage: ``no effort is put into optimizing the code for speed" \cite{ECDARNET}.
In contrast to this, the Reveaal engine is intended to be fast and parallelizable. 
Recently, the libraries which the engine uses have been rewritten in Rust from C/C++, with the intention of implementing multithreading. 

ECDAR's testing framework is written in Kotlin. 
The testing framework uses a collection of test cases to test both of the engines. 
The testing framework is vital to perform conformance testing between j-Ecdar and Reveaal as well as automated performance testing. 