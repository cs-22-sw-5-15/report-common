\subsection{Model Checking in ECDAR}
ECDAR currently combines a graphical user interface for modelling real-time systems using timed input/output automata with a model checking engine. Currently ECDAR has two verification engines: j-ecdar and Reveaal, that is written in java and rust respectively.

ECDAR takes queries, that it uses to check components up to a specification. 

ECDAR combines componenents, using operators and checks if it satisfies a specifications.

\todo{more about ecdar}

%To sum up, what the purpose of ECDAR is:
%\begin{quote}
%"[to integrate] conformance testing into a new IDE that now features modelling, verification, and testing. The new tool uses model-based mutation testing, requiring only the model and the system under test, to locate faults and to prove the absence of certain types of faults." \cite{Gundersen_2018}
%\end{quote}


This section will go over the ECDAR tool, to further illustrate the intent of ECDAR and how far it has been developed. The GUI of ECDAR 2.3.3 can be seen in \autoref{fig:ECDAR-gui}.

\begin{table}[H]
\begin{tabular}{ll}
\textbf{Query Type} & \textbf{Description} \\
reachability [E$<>$]     & x                    \\
refinement [$\leq$]      & x                    \\
quotient [\textbackslash]& x                    \\
specification [Spec]     & x                    \\
implementation [Imp]     & x                    \\
consistency [lCon]       & x                    \\
global-consistency [gCon]& x                    \\
bisim [bsim]             & x                    \\
get-component [get]      & x                    \\
\end{tabular}
\caption{\label{tab:querytypes}Table describing the different query types. The functionality of every query type is currently not implemented.}
\end{table}