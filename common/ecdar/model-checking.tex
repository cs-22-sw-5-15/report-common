\subsection{Model Checking}

Systems nowadays are big and complex and consists of multiple components designed by independent teams \cite{ecdartheory}. Model checking is a method used in the software industry for years as a form of computer aided verification where logic is used for bug finding \cite{modelchecking_handbook}. This formal verification technique determines whether given properties of a system are satisfied by a model. A model is a system represented using mathematics. This could for example be finite state-systems. 

To illustrate the importance of model checking, consider the following example:

Imagine for a moment that you are working for a space agency and your mission is to send a satellite into space.
You have spent several months testing the satellite to make sure it is correct and everything is working as intended.
All tests have passed for every edge case your engineers could think of.
The satellite is ready and for launch.
The day after launch however, a major problem in the system is found. 
It will need to be fixed if the satellite is to have any use, but doing so might be extremely expensive or outright impossible.

As the model grows in size, it becomes ever more complex to verify by hand.
The immense complexity of modern systems dictates that the verification must be computer aided in order to be feasible \cite{modelchecking_handbook}. ECDAR is one such tool

