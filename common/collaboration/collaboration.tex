\section{Collaboration}\label{sec:cooperation}
\textit{The following sections is written in collaboration with other ECDAR project groups.}


%Introduction to the concept of this semester project and the fact that we are continuing development of an existing program:
This semester project specifically focuses on multi-project cooperation, which means that several project groups will be cooperating on the same project. In total, six project groups consisting of six members each will be collaborating to further extend ECDAR. Five of these groups will be focusing on further development of the Reveaal engine, while the last group will focus on development of the graphical user interface of ECDAR. However, all six groups....


The ECDAR project is a newly started project and therefore there have not been any iteration of the project.
For the first iteration the groups first job was to startup everything collaborative related. 
The way the groups could uphold knowledge sharing was by planning scrum meetings.

There were six project groups that chose ECDAR as their semester project. 
Five groups that choose to work with the Reveaal engine and one group that choose the graphical user interface engine.
%Because there were so many students working on the ECDAR project the scrum meetings would get too big if every group member attended the meetings.
%Based on that the groups choose a committee to attend the scrum meetings.


%Har tilføjet det her -Christian

